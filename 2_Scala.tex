
\section{Scala}

\begin{breakbox}
\textbf{Expression Problem:}\\
Wie kann man zu einem Programm \\
a) neue Datentypen \\
b) neue Operationen \\
hinzufuegen, ohne\\
1) bestehenden Code anzupassen\\
2) keine Typensicherheit zu verlieren \\\\
\emph{Loesung OO}
\begin{itemize}
	\pro neue Datentypen mit Vererbung
	\con neue Operationen, schwer da Hierarchie anzupassen. Loesung: Visitor
\end{itemize}
\emph{Loesung FP}
\begin{itemize}
	\con neue Datentypen, schwer alle Pattern-Matches in Fkt anpassen. Loesung Haskell: Type Classes
	\pro neue Operationen, einfach da getrennt von Daten
\end{itemize}
\end{breakbox}

\begin{breakbox}
\textbf{Types:}\\
\emph{Typisierung:} statisch mit type-inference\\
\emph{scala.Any:} \mintinline[fontsize=\scriptsize]{scala}{AnyVal, Anyref}\\
\emph{scala.AnyVal:} \mintinline[fontsize=\scriptsize]{scala}{Int, Float, Boolean, Nothing}\\
\emph{scala.AnyRef:} \mintinline[fontsize=\scriptsize]{scala}{String, List, Null, Nothing}\\
\end{breakbox}

\begin{breakbox}
\textbf{List and Tupel:}\\
\emph{List:} Kein Interface, konkrete Klasse, immutable, var. Anz. desselben Typs\\
\begin{scalacode}
val aList = List(2,3,4) // aList: List[Int]
val list = 1 :: 2 :: 3 :: Nil
list.head //1
list.tail //List(2,3)
list(0) //1
\end{scalacode}

\emph{Tupel:} Fixe Anz. Elemente unterschiedlichen Typs\\
\begin{scalacode}
val tupel2 = Tuple2(1, "String") //tupel2: (Int, String)
val theInt = tupel2._1
val triple = (1,2,"3") //triple: (Int, Int, String)
\end{scalacode}
\end{breakbox}

\begin{breakbox}
\textbf{Variablen und Kontrollstrukturen:}\\
\begin{scalacode}
val i = 42 //konstante
var j = 43 //variable
lazy val k = 44 //init bei erstem Zugriff
val max = if(a>b) a else b //ifelse als expression
\end{scalacode}
\end{breakbox}

\begin{breakbox}
\textbf{Pattern Matching:}\\
\begin{scalacode}
//example with Int
def patternMatching(i:Int) = {
  i match {
    case 0 => "Null"
    case 1 => "Eins"
    case _ => "?"
  }
}

//example with case class
case class Person(name:String, alter:Int)
def matchAPerson(person: Person) = {
  person match {
    case Person("Peter", 42)           => "Found Peter!"
    case Person(name, age) if age < 18 => s"minor \$name"
    case p if p.alter == 42   => "Person with a magical age!"
    case Person(name, _)               => s"adult \$name"
    //case _                             => "unknown person"
  }
}

val ruedi = Person("Ruedi", 42)
matchAPerson(ruedi) //person with magical age!
\end{scalacode}
\end{breakbox}

\begin{breakbox}
\textbf{Methoden:}\\
\begin{scalacode}
def add1(m:Int, n:Int):Int = {m+n}
def add2(m:Int, n:Int = 1) = m+n
def add(m:Int)(n:Int) = m+n

add1(10,5)
add2(1)
add1(n=10, m=5)
add(10)(5)
val addToTen = add(10) _ //Partial application e.g. currying
\end{scalacode}

\emph{Call-by-value / Call-by-name}\\
\begin{scalacode}
//call by value
def unless(condition:Boolean, then:Unit) = {
	if(!condition) { then }
}
// exception thrown cause all params are
// evaluated from left to right
unless(1+2==3, throw new RuntimeException()) 

//call by name
def unless(condition:Boolean, then:=>Unit) = {/*...*/}
// then param only evaluated if used
// then parameter is a function with no param 
// and Unit as return type
\end{scalacode}
\end{breakbox}

\begin{breakbox}
\textbf{Functions:}\\
\begin{scalacode}
(m:Int, n:Int) => m+n
(m:Int, n:Int) => {m+n}
(m:Int) => (n:Int) => m+n //currying

// 3 main styles
//func1: Int => Int = <function1>
val func1: (Int) => Int = (x:Int) => x^2
//func2: Int => Int = <function1>
val func2 : Function1[Int, Int] = (x:Int) => x^2
//func3: Function[Int,Int] = <function1>
val func3 = new Function[Int, Int] { def apply(x:Int):Int = x^2  }

//inplace
case class Person(name:String, age:Int)
val people = Person("Josef", 21) :: Person("Maria", 17) :: Nil
val isAdult = (p:Person) => p.age >= 18 //predicate
// def filter(p: (A) => Boolean): List[A]
var adults = people.filter(isAdult)
adults = people.filter((p:Person) => p.age >= 18)
adults = people.filter(p => p.age >= 18)
//block allows multiple statements
adults = people.filter{p => p.age >= 18 && p.age < 99}

//function companion object
val isAdult2 = new Function[Person, Boolean] {
	def apply(p:Person) : Boolean = p.age >= 18
}
\end{scalacode}
\end{breakbox}

\begin{breakbox}
\textbf{Traits:}\\
\begin{scalacode}
trait Equal[T] {
  def isEqual(t:T):Boolean
  def isNotEqual(t:T) = !isEqual(t)
}
trait Named { def named:String }

//this mixin with uap is really smooth shit!
case class User(named:String) extends Equal[User] with Named {
  def isEqual(t:User):Boolean = this.named == t.named
}
\end{scalacode}
\end{breakbox}

\begin{breakbox}
\textbf{Klassen:}
\begin{itemize}
	\item abstrakte und finale Klassen
	\item Einfachvererbung
	\item Jeder Klasse hat primaeren Ctor, der immer aufgerufen wird\\
\end{itemize}
\emph{Getter and Setter:}
\begin{itemize}
	\item keine public Datenfelder
	\item Zugriff ueber getter (\mintinline[fontsize=\scriptsize]{scala}{p.age}) und setter (\mintinline[fontsize=\scriptsize]{scala}{p.age = 42})
	\item \underline{Uniform Access Principle:} Aufrufer weiss nicht ob property oder methode
\end{itemize}
\begin{scalacode}
abstract class Person(name:String, age:Int) {
  def jobDescription : String //abstract
  override def toString() = s"name: \$name, age: \$age"
}

class Student(val name:String, age:Int, private val sid:Int) 
		extends Person(name, age) {
  def this(name:String, age:Int, sid:Int, something:String) 
  		= this(name, age, sid) //sec ctor
  override def jobDescription = "Student"
  val readOnlyVal = 42
  var readWriteVar = 43
  private val privateVal = 44
  def aMethod(f:Int=>Int):Int = {f(42)}
}

val student = new Student("Pamela", 19, 696969)
student.readWriteVar= 44
student.aMethod(x=>x*x)
\end{scalacode}
\end{breakbox}

\begin{breakbox}
\textbf{Case Classes:}\\
\mintinline[fontsize=\footnotesize]{scala}{case class Person(alter:int, name:String)}
\begin{itemize}
	\item Getter fuer alle Ctor-Param
	\item \mintinline[fontsize=\scriptsize]{scala}{toString, equals, hashCode} automatisch
	\item Copy-Methode \mintinline[fontsize=\scriptsize]{scala}{val p = person.copy(age=person.age+5)}
	\item Verwendbar im Pattern-Matching
	\item kein new noetig \mintinline[fontsize=\scriptsize]{scala}{val p = Person("X",12)}
\end{itemize}
\end{breakbox}

\begin{breakbox}
\textbf{Singleton Objects:}\\
\begin{itemize}
	\item Scala hat kein stat. Methoden
	\item SO werden autom. instanziert (kein ctor)
	\item Heisst das Object gleich wie die Klasse spricht man vom \underline{Companion Object}
	\item Bei Case-Classen generiert Comp Companion Object mit apply-Methode (\underline{Factory!})
\end{itemize}
\begin{scalacode}
object Person { //ohne ctor
	def apply(name:String) = new Person(name)
}
Person("Mirko") // entspricht Person.apply("")
\end{scalacode}
\end{breakbox}

\begin{breakbox}
\textbf{Scala ADT:}\\
\begin{scalacode}
// FP Styled, for OO-Style include methods in class
sealed class Lst[T]()
case class Nil[T]() extends Lst[T]
case class Cons[T](head:T, tail:Lst[T]) extends Lst[T]

object Lst {
  def length[T](lst:Lst[T]):Int = {
    lst match {
      case Nil() => 0
      case Cons(_, tail) => 1+length(tail)
    }
  }
}

val list:Lst[Int] = Cons(1, Cons(2, Cons(3, Nil())))
Lst.length(list)
\end{scalacode}
\end{breakbox}

\begin{breakbox}
\textbf{Generics with type-bounds:}
\begin{itemize}
	\item \mintinline[fontsize=\scriptsize]{scala}{<:} extends  
	\item \mintinline[fontsize=\scriptsize]{scala}{>:} super 
\end{itemize}
\begin{scalacode}
case class Node[T <: Comparable[T] ](value:T, 
	left:Option[Node[T]], right:Option[Node[T]])

// selfmade using
def using[T <: Closeable](resource : => T)(fun: => Unit) = {
	try{fun(resource)} finally {resource.close()}
}
\end{scalacode}
\end{breakbox}

\begin{breakbox}
\textbf{Varianz:}\\
\begin{scalacode}
// Covariant: auch Subtypen erlaubt 
// (Return values are in a covariant position)
anAnimal = getGiraffe() -- covariante Umwandlung
// Contravariant: auch Supertypen erlaubt 
// (Parameter are in contravariant position)
passAnimal(aGiraffe) -- contravariante Umwandlung

--Baskets
class CoVariantBasket[+T](init:T) {
  private val current : T = init
  def get : T = current
  //def set(t:T) = current = t
}
class ContraVariantBasket[-T](var init:T) {
  //private var current : T = init //now allowed
  //def get : T = current //now allowed
  def set(t:T) = init = t
}

val bananaBasket : CoVariantBasket[Banana] = 
	new CoVariantBasket[Banana](new Banana)
val fruitBasket : CoVariantBasket[Fruit] = bananaBasket

val fruitBasket2 : ContraVariantBasket[Fruit] = 
	new ContraVariantBasket[Fruit](new Apple)
val bananaBasket2 : ContraVariantBasket[Banana] = fruitBasket2
\end{scalacode}
\end{breakbox}


