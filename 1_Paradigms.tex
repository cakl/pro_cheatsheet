
\section{Paradigms}

\begin{breakbox}
\textbf{Imperative:}
\begin{itemize}
	\setlength{\itemsep}{0pt}
    \setlength{\parskip}{0pt}
    \setlength{\parsep}{0pt}
    \setlength{\itemindent}{-0.15in}
	\item How program operates
	\item Statements, change a programs state
	\item Program = Seq. of statements
	\item Computation = Change of state
\end{itemize}
\textbf{Declarative:}
\begin{itemize}
	\setlength{\itemsep}{0pt}
    \setlength{\parskip}{0pt}
    \setlength{\parsep}{0pt}
    \setlength{\itemindent}{-0.15in}
	\item Logic of computation without describing its control flow
	\item No side effects
	\item Program = theories of formal logic
	\item Computation = deductions
\end{itemize}
\end{breakbox}

\begin{breakbox}
\textbf{Logic Programming:}
\begin{mdframed}
  \begin{center}
  	Computation: Proof of Predicate 
  \end{center}
\end{mdframed}

$mensch(sokrates)$ \\
$\forall mensch(x) \implies sterblich(x)$ \\
$sterblich(sokrates)$\\

\textbf{Prolog:}
\vspace{-5mm}
\begin{minted}[mathescape, numbersep=5pt, frame=lines,framesep=2mm, fontsize=\scriptsize]{prolog}
mensch(sokrates).
sterblich(X):-mensch(X).
?-sterblich(X) % X = sokrates
\end{minted}
\textbf{Prolog Syntax:}\\
\emph{Fakten:} \mintinline{prolog}{bhf(bern)}\\
\emph{Variable (X)}: \mintinline{prolog}{verb(X,Y)}\\
\emph{Regeln (Head:-Body):}\\ \mintinline{prolog}{verb(X,Y) :- bhf(X),bhf(Y)}\\
Head gilt, wenn \underline{alle} Bedingung des Body wahr sind\\
\emph{Lists:}\mintinline{prolog}{[a,b,c] [], [a|[b,c]]}\\

\textbf{Unifikation:}\\
$f(a,X,g(X)) \implies f(a,b,Y)$ 
$\text{with: } \{X/b, Y/g(b)\}$\\

\textbf{Prolog Example:}\\
%\vspace{-5mm}
\begin{prologcode}
% LISTS
% [1,2,3] = [1,2,3].
% [1,2,3] = [1|[2,3]].
% [1,2,3] = [1,2,3|[]].

member(M,[M|_]).
member(M,[_|T]) :- member(M,T).
% member(a,[a,b,c,d]).
% member(X,[a,b,c,d]).
% member(a,X).

append([],Y,Y).
append([H|X],Y,[H|Z]) :- append(X,Y,Z).

perm([],[]).
perm(L,[H|T]) :- append(V,[H|U],L), append(V,U,W), perm(W,T).

% perm([1,2,3],X).

sorted([]).
sorted([_]).
sorted([H1,H2|T]) :- H1 =< H2, sorted([H2|T]).

% sorted([1,2,3,4]).
% sorted([1,2,30,4]).

n_sort(L,SL) :- perm(L,SL), sorted(SL).

% n_sort([2,1],X).
% n_sort([2,1,0,5,3,4],X).

% zuege
verbindung(StadtA, StadtZ, GAbfZ, Wie) :-
verbindung(StadtA, StadtZ, GAbfZ, [StadtA], Wie).
verbindung(StadtA, StadtA, GAbfZ, S, []).
verbindung(StadtA, StadtZ, GAbfZ, S, [(Num,StadtT)|Wie]):-
zug(StadtA, StadtT, AbfZ, AnkZ, Num),
\+ member(StadtT, S),
AbfZ >= GAbfZ,
verbindung(StadtT, StadtZ, AnkZ, [StadtT|S], Wie).
\end{prologcode}
\end{breakbox}

\begin{breakbox}
\textbf{Functional Programming:}
\begin{mdframed}
  \begin{center}
  	Computation: Evaluation of mathematical Functions
  \end{center}
\end{mdframed}
\emph{Why declarative?}
\begin{itemize}
	\setlength{\itemsep}{0pt}
    \setlength{\parskip}{0pt}
    \setlength{\parsep}{0pt}
    \setlength{\itemindent}{-0.15in}
	\item Programming done with expression
	\item Foundation: lambda calculus
	\item no (or controlled) mutable state
\end{itemize}
\emph{Lambda calculus:} Formalisation of the concept of a math. function. Function as a Relation between 1 input and \underline{exactly} 1 output \\\\
\emph{No mutable state:}
\begin{itemize}
	\setlength{\itemsep}{0pt}
    \setlength{\parskip}{0pt}
    \setlength{\parsep}{0pt}
    \setlength{\itemindent}{-0.15in}
	\item \underline{Referential Transparency} $f(x)$ only depends on the definition of $f$ and the value $x$
	\item No mutable variables
	\item No assignements
	\item No imperative control structures
	\item All data structures are immutable
\end{itemize}
\emph{Functions as first-class citizen:}
\begin{itemize}
	\setlength{\itemsep}{0pt}
    \setlength{\parskip}{0pt}
    \setlength{\parsep}{0pt}
    \setlength{\itemindent}{-0.15in}
	\item Funcs can be anonymous ($\lambda x.x+1$)
	\item Funcs can be the input/output to other \emph{higher-order} Functions
	\item Funcs can be composed ($f \circ g$)
\end{itemize}
\end{breakbox}

\begin{breakbox}
\textbf{Expression Evaluation Stategies:}
\emph{Call by Value}
\begin{itemize}
	\setlength{\itemsep}{0pt}
    \setlength{\parskip}{0pt}
    \setlength{\parsep}{0pt}
    \setlength{\itemindent}{-0.15in}
	\pro Params evaluated only once
	\con All Params evaluated (even unused)
	\con May not terminate
\end{itemize}
\emph{Call by Name}
\begin{itemize}
	\setlength{\itemsep}{0pt}
    \setlength{\parskip}{0pt}
    \setlength{\parsep}{0pt}
    \setlength{\itemindent}{-0.15in}
	\con Params evaluated multiple times
	\pro Unused Params not evaluated
	\pro Terminates if there is a terminating strategy (e.g. base case)
\end{itemize}
\emph{Never-terminating-strategy:} recursion without base case \\
\emph{Terminates-by-name:} \mintinline{haskell}{if true then o else infloop}\\
\emph{Haskell's evaluation strategy:}\\\underline{Lazy Evaluation}, call-by-name and rember past evaluations
\end{breakbox}

\begin{breakbox}
\textbf{Haskell ADTs:}\\
\emph{With type parameters}
\vspace{-5mm}
\begin{minted}[mathescape, numbersep=5pt, frame=lines,framesep=2mm, fontsize=\scriptsize]{haskell}
data Lst a = Nil | Cons a (Lst a)
Lst :: * -> * 		-- type ctor
Nil :: Lst a 		 -- term
Cons :: a -> Lst a -> Lst a   -- term

data Maybe a = Nothing | Just a
Maybe :: * -> *
Nothing :: Maybe a
Just :: a -> Maybe a
\end{minted}
\emph{Without type parameters}
\vspace{-5mm}
\begin{minted}[mathescape, numbersep=5pt, frame=lines,framesep=2mm, fontsize=\scriptsize]{haskell}
data Bool = True | False --type
True :: Bool  -- term
False :: Bool -- term
\end{minted}
\end{breakbox}

\begin{breakbox}
\textbf{Haskell ADT Examples}
\begin{haskellcode}
data Lst a = Nil | Cons a (Lst a) deriving (Show, Eq)

--len :: Num a => Lst t -> a
len Nil = 0
len (Cons hd tl) = 1 + len tl

--mem :: Eq a => a -> Lst a -> Bool
mem e Nil = False
mem e (Cons hd tl) = (e == hd) || (mem e tl)

--appd
appd e Nil = Cons e Nil
appd e (Cons hd tl) = Cons hd (appd e tl)

--nth
nth :: Int -> Lst a -> Maybe a
nth n Nil = Nothing
nth n (Cons xs x) | n == 1 = (Just xs)
                  | otherwise = nth (n-1) x

--prod
prod :: Num a => Lst a -> a
prod Nil = 1
prod (Cons hd tl) = hd * prod tl

--disj --conj: replace || with &&
disj :: Lst Bool -> Bool
disj Nil = False
disj (Cons hd tl) = hd || (disj tl)

--mapp
mapp :: (a -> b) -> Lst a -> Lst b
mapp f Nil = Nil
mapp f (Cons hd tl) = Cons (f hd) (mapp f tl)

--filter
fltr :: (a -> Bool) -> Lst a -> Lst a
fltr f Nil = Nil
fltr f (Cons hd tl) = if (f hd) then 
			(Cons hd (fltr f tl)) else (fltr f tl)

--reduce 
reduce f i Nil = i
reduce f i (Cons hd tl) = f hd (reduce f i tl)

--redtotal
redTotal :: Num a => Lst a -> a
redTotal lst = reduce (\x y -> x + y) 0 lst

--redDisj
redDisj :: Lst Bool -> Bool
redDisj lst = reduce (\x y -> x || y) False lst

--redApp
redApp :: Lst a -> Lst a -> Lst a
redApp lst1 lst2 = reduce (\ x y -> Cons x y) lst2 lst1

--insert
insert :: Ord a => a -> Lst a -> Lst a
insert x Nil = Cons x Nil
insert x (Cons hd tl) | x < hd = (Cons x (Cons hd tl))
                      | otherwise = (Cons hd (insert x tl))

--insertionsort
isort :: Ord a => Lst a -> Lst a -> Lst a
isort lst Nil = lst
isort sorted (Cons hd tl) = isort (insert hd sorted) tl

--quicksort
partition :: (a -> Bool) -> Lst a -> (Lst a, Lst a)
partition f ls = (fltr f ls, fltr (\x -> not (f x)) ls)
app Nil ls = ls
app (Cons hd tl) ls = Cons hd (app tl ls)

quicksort Nil = Nil
quicksort (Cons hd tl) = app (quicksort smaller) 
			(Cons hd (quicksort larger))
  where
    (smaller, larger) = (partition (\x -> x < hd) tl)
\end{haskellcode}
\end{breakbox}

\begin{breakbox}
\textbf{Haskell Tree ADT:}\\
\begin{haskellcode}
data Tree a = Nil | Node a (Tree a) (Tree a) 
			deriving(Show, Read, Eq)

--treeMap
treeMap :: (a -> b) -> Tree a -> Tree b
treeMap f Nil = Nil
treeMap f (Node v l r) = Node (f v) (treeMap f l) (treeMap f r)

--repTree
repTree :: (Ord a, Num a) => a1 -> a -> Tree a1
repTree e depth 
    | depth <= 0 = Nil
    | otherwise  = Node e (repTree e (depth-1)) 
    			(repTree e (depth-1))

--inOrder, for preorder/postoder change place of [v]
inOrder :: Tree a -> [a]
inOrder Nil = []
inOrder (Node v l r) = (inOrder l) ++ [v] ++ (inOrder r)

--treeInsert
treeInsert :: Ord a => a -> Tree a -> Tree a
treeInsert x Nil = Node x Nil Nil
treeInsert x (Node v l r) | x <= v = Node v (treeInsert x l) (r) 
                          | otherwise = Node v l (treeInsert x r)

--treeSearch                          
treeSearch :: (Ord a) => a -> Tree a -> Bool  
treeSearch x Nil = False  
treeSearch x (Node a left right)  
    | x == a = True  
    | x < a  = treeSearch x left  
    | x > a  = treeSearch x right

--treeSort
treeSort ls = inOrder (foldr treeInsert Nil ls)
\end{haskellcode}
\end{breakbox}
